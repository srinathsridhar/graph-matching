\section{Realistic Models of Recommendation Graphs}
%We realize that the models described previously are a bit
%theoretical. 
Even though we studies the fixed-degree model in detail, recommendation systems based on relevance in practice
will not have edges that are spread uniformly at random. Items that are
about specific topics are much more likely to interlink within
themselves than to those outside that topic, leading to
clusters of recommendations.

To understand this substructure in underlying
graphs in practice, we compiled results from several e-commerce retailers that
have been aggregated and anonymized in the table shown below. For each
retailer, we compiled the product ontology present within the
site that places a product in this tree-like categorization. E.g., a
juicer called ``Breville Juice Fountain Plus'' is in the tree path:
Home $\rightarrow$ Juicers $\rightarrow$ High Speed Juicers
$\rightarrow$ Breville Juice Fountain Plus. We then examined the
recommendations from products at different depths of the hierarchy. In
the table in Figure~\ref{fig:hier} we examined the edges adjacent to
products at depth 4 or greater. We calculated the percentage of edges
connecting to products that had different least common ancestors (LCA) with the current product.  We
then randomized the edges so that we can compare how the graph would
have looked if there was no substructure and re-calculate the
distribution of the edges and the LCA levels. We noticed that the
uniform distribution had edges that had very shallow LCA indicating that 
most edges did not follow the product hierarchy while in reality,
the endpoints of edges recommended had much deeper LCA meaning recommendation edges were clustered based on the product hierarchy. This led us
to formalize this new model of input graphs that we study in Subsection~\ref{hierarchy}
as the {\em hierarchical tree model}.

\begin{figure}[h]
  \centering
  \begin{tabular}{ |c|c|c|c|c|c|c|c|c|c| }
    \hline
    $LCA Level$ & 0 & 1 & 2 & 3 & 4 & 5 & 6 & 7 \\ \hline
    Uniform & 13.4 & 69.7 & 12.5 & 2.6 & 1.2 & 0.6 & 0.0 & 0.0 \\ \hline
    Hierarchical & 7.1 & 1.9 & 8.0 & 24.9 & 52.3 & 5.5 & 0.2 & 0.1\\
    \hline
  \end{tabular}
  \caption{Percentage of edges based on different LCA levels of their endpoints, based on the product ontologies of different merchants and as generated by the uniform model.}\label{fig:hier}
\end{figure}

In a second analysis, we simply truncated the product hierarchy
at depth 3 and collected the list of disjoint clusters in the hierarchy. We
then examined all the recommendations and partitioned them into those going between each pair of these clusters. In a uniform distribution, we would expect the
edges to be equally likely to span across each pair of clusters
(assuming clusters are equal sized). But what we observed was that
different pairs of clusters had different edge-densities. For instance, an Espresso Machine might point more to other Coffee Machines or
Coffee Beans (note that Coffee Beans and Espresso Machine might share
no LCA apart from the root) than to other clusters. These results are
shown in Figure~\ref{fig:hierarchy} that clearly demonstrates that these
clusters exist and have different densities than the uniform sampling model. This
motivated us to define and study the {\em cartesian product model} in
Subsection~\ref{cartesian} which is orthogonal to the uniform and
hierarchical tree models. Finally, in Section~\ref{weighted}, we study the {\em weighted model}
which assigns weights to the graph edges so that we can incorporate traffic
patterns across a website besides just relevance-based recommendations.

\subsection{Hierarchical Tree Model}
\label{hierarchy}

We assume that we are given a bipartite graph $G=(L,R,E)$. The vertex sets $L$
and $R$ are the leaf sets of two full binary trees $T_L$ and $T_R$ of
depth $D$ where there is a one-to-one correspondence between the
subtrees of these two trees. We also assume that each branching in
both $T_L$ and $T_R$ splits the nodes evenly into the two
subtrees. As in the previous sections, we set $|L|/|R|=k$,
and require that this ratio is still $k$ if we take any subtree on the left
and its corresponding subtree on the right. For simplicity of notations, we
will use a subtree and its leaf set interchangeably. We assume that the trees are
fixed in advance but $G$ is generated probabilistically according to
the following procedure. Let $u\in L$ and $T_L^0, \ldots T^{D-1}_L$ be
the subtrees it belongs at depths $0,\ldots, D-1$. Also, let
$T_R^0,\ldots, T_R^{D-1}$ be the subtrees on the right that correspond
to these trees on the left. We let $u$ make an edge to $d_{D-1}$ of
the vertices in $T_{R}^{D-1}$, $d_{D-2}$ edges to the vertices in
$T_{R}^{D-2} \backslash T_{R}^{D-1}$ and so on. The $d_i$ edges out of $u$ are chosen uniformly from $T_R^i \setminus T_R^{i-1}$ 
and let $d = d_{0} + \ldots + d_{D-1}$.\vs

\begin{figure}[h]
\centering
\includegraphics[width=0.6\textwidth]{images/hierarchy_tree.png}
\begin{minipage}[h]{0.7\textwidth}
\caption{This diagram shows the notation we use for this model and the 1-to-1 correspondance of subtrees}\label{fig:hierarchy}
\end{minipage}
\end{figure}

Our goal now is to find a variant of matching~\cite{Gabow1983} in this
graph that is close to optimal in expectation. That is, our degree
upper and lower bounds on vertices in $L$ and $R$ are $c$ and 1
respectively. Let $c = c_0 + \ldots + c_{D-1}$ be similar to how we
defined $d$.  To combine the analysis of the randomness of the
algorithm and the randomness of the graph, the algorithm will pick
$c_{i}$ edges uniformly from among the $d_{i}$ edges going to each
level of the subtree. This enables us to think of the subgraph our
algorithm finds as being generated by the identical graph generation
process, but with fewer neighbors selected. With this model and
parameters in place, we can have the following analog of our main
theorem for $a=1$ for the hierarchical model and is proved in Appendix~\ref{sec:appendix}.

\begin{thm}
Let $S$ be the subset of edges $v\in R$ such that $\deg_H(v) \geq 1$. Then
\[ \E[S] \geq r(1-\exp(-ck)) \]
\end{thm}

%\begin{proof}
%Let $v\in R$ and let $T_L^{D-1}, T_L^{D-2}\backslash T_L^{D-1},
%\ldots, T_L^0\backslash T_L^1$ be the sets it can take edges
%from. Since $T_L$ and $T_R$ split perfectly evenly at each node the
%vertices in these sets will be chosen from $r_{D-1}, r_{D-1},
%r_{D-2},\ldots, r_{1}$ vertices in $R$ as neighbors,
%where $r_i$ is the size of subtree of the right tree rooted at depth
%$i$. Furthermore, each of these sets described above have size
%$l_{D-1}, l_{D-1}, l_{D-2}, \ldots, l_{1}$ respectively, where $l_i$
%is the size of a subtree of $T_L$ rooted at depth $i$. It follows that
%the probability that $v$ does not receive any edges at all is at most
%
%\begin{align*}
%	      \Pr[\lnot X_v]
%	&=    \left(1-\frac{1}{r_{D-1}}\right)^{c_0l_{D-1}}\prod_{i=1}^{D-1}\left(1 - \frac{1}{r_i}\right)^{c_{D-i} l_i} \\
%	&\leq \exp\left(-\frac{l_{D-1}}{r_{D-1}}c_0\right)\prod_{i=1}^{D-1} \exp\left(-\frac{l_i}{r_i}c_{D-i}\right) \\
%	&=    \exp\left(-(c_0 + \ldots + c_{D-1})k\right) \\
%	&=    \exp(-ck)
%\end{align*}
%
%Since this is an indicator variable, it follows that
%\[ \E[S] = \E\left[\sum_{v \in R} X_v \right] \geq r \left(1-\exp(-ck)\right) \]
%\end{proof}

Note that this is the same result as we obtained for the fixed degree
model in Section \ref{fixed-degree}. In fact, the approximation
guarantees when $ck \ll 1$ or $ck \gg 1$ hold exactly as before.\vs

The sampling of $H$ can be done algorithmically because we separated
out the edge generation process at a given depth from the edge
generation process at deeper subtrees. There is no ambiguity as to why
an edge is in the underlying graph. That is, if we superimpose $T_L$
and $T_R$, then an edge between $u_l\in L$ and $v_r\in R$ must have
come from an edge generated by the process corresponding to the lowest common ancestor of $u_l$ and $v_r$. This way, the algorithm can actually sample intelligently and in the same way that the graph was generated in the first place. Also note
that we do not have to assume that the trees $T_L$ and $T_R$ are
binary. We only need the trees to be regular and evenly divided at
each vertex since the proof only relies on the proportions of the
sizes of the subtrees in $T_L$ and $T_R$.


\subsection{Cartesian Product Model}
\label{cartesian}
We can see another shortcoming of the uniform graph model if we think
of products as being clustered into categories. This is similar to the
hierarchical tree mode, but our assumptions are weaker in that we don't
mandate a complete hierarchy and that the clusters can be unrelated. In
such a model, we would expect many recommendation edges to be
intra-cluster edges rather than inter-cluster edges. Assuming that
cluster sizes are the same, the uniform model would generate the same
number of edges between any two pairs of clusters. However, in an
aggregate of data collected from real merchants, we can see that some
pairs of clusters have many more edges between them than the uniform
model would predict:

\begin{figure}[h]
\centering
\includegraphics[width=0.6\textwidth]{images/cartesian_histogram.png}
\begin{minipage}[h]{0.7\textwidth}
\caption{Histogram showing the percentage edges between pairs of clusters. The long tail is omitted.}
\end{minipage}
\end{figure}

This finding motivates the following model. We assume that
$L$ has been partitioned into $t$ subsets $L_1,\ldots, L_t$ and
that $R$ has been partitioned into $t'$ subsets $R_1,\ldots,
R_{t'}$. For convenience, we let $|L_i| = l_i$ and $|R_i|=r_i$. Given
this suppose that for each $1\leq i\leq t$ and each $1\leq j\leq t'$,
$G[L_i, R_j]$ is an instance of the fixed degree model with
$d=d_{ij}$. We assume that for all $i$, we have $\sum_{j=1}^{t'}
d_{ij} = d$ for some fixed $d$. Also assume that we have fixed in
advance $c_{ij}$ for each $1\leq i\leq t$ and $1\leq j\leq t'$ that
satisfy $\sum_{j=1}^{t'} c_{ij} = c$ for all $i$ for some fixed $c$.
To sample $H$ from $G$, we sample $c_{ij}$ neighbors for each
$u_i\in L_i$ from $R_i$. Letting $S$ be the set of vertices in
$v\in R$ that satisfy $\deg_H(v)\geq 1$, we can show the following theorem
that is proved in Appendix~\ref{sec:appendix}.

\begin{thm}
With $S$, $G$ and $H$ defined as above, we have
\[ \E[S] \geq r - \sum_{j=1}^{t'} r_j \exp\left(-\sum_{i=1}^t c_{ij} \frac{l_i}{r_j}\right)\]
where the expectation is over $G$ and $H$.
\end{thm}
%\begin{proof}
%Let $v_j \in R_j$ be an arbitrary vertex and let $X_{v_j}$ be the
%indicator variable for the event that $\deg_H(v_i) \geq 1$. The
%probability that none of the neighbors of some $u_i\in R_i$ is $v_j$
%is exactly $(1-\frac{1}{r_j})^{c_{ij}}$. It follows that the
%probability that the degree of $v_j$ in the subgraph $H[L_i,R_j]$ is 0
%is at most $(1-\frac{1}{r_j})^{c_{ij}l_i}$. Considering this
%probability over all $R_j$ gives us:
%\[ \Pr[X_{v_i} = 0] = \prod_{i=1}^{t} \left(1-\frac{1}{r_j}\right)^{c_{ij} l_i} \leq \exp\left(-\sum_{i=1}^t c_{ij} \frac{l_i}{r_j}\right)\]
%
%By linearity of expectation $\E[S] = \sum_{i=1}^{t'} r_i \E[X_{v_i}]$,
%so it follows that
%\[ \E[S] \geq \sum_{j=1}^{t'} r_j \left(1-\exp\left(-\sum_{i=1}^t c_{ij} \frac{l_i}{r_j}\right)\right) = r - \sum_{j=1}^{t'} r_j \exp\left(-\sum_{i=1}^t c_{ij} \frac{l_i}{r_j}\right)\]
%\end{proof}

This model is interesting because it can capture a broader set of
recommendation subgraphs than the fixed degree model. However, it is
difficult to estimate how good a solution will be without knowing
the sizes of the sets in the partitions. We note that we
obtain the approximation guarantee of $(1-\exp(-ck))$ provided that
$l_i/r_j = k$ for all $i$ and $j$ where $k$ is some fixed
constant. Another interesting point about this model and the algorithm
we described for sampling $H$ is that we are free to set the $c_{ij}$
as we see fit. In particular, $c_{ij}$ can be chosen to maximize the
approximation guarantee in expectation we obtained above using
gradient descent or other first order methods prior to running the
recommendation algorithm to increases the quality of the solution.


\subsection{Weighted Model}
\label{weighted}
The fixed degree model of Section \ref{fixed-degree} is a simple and
convenient model, but the assumption that all recommendations hold the
same weight is unrealistic. This motivates fixing the graph to be the
complete bipartite graph $K_{l,r}$, and giving the edges i.i.d weights
with mean $\mu$. We modify the objective function accordingly, so that
we count only the vertices in $R$ which have weight $\geq 1$. If we
assume that $ck\mu \geq 1+\epsilon$ for some $\epsilon > 0$, then
the naive sampling solution we outlined in Section \ref{fixed-degree}
still performs exceptionally well. If we let $S$ be the size of the
solution produced by this algorithm. We have the following theorem that is
proved in Appendix~\ref{sec:appendix}.

\begin{thm}
Let $G=K_{l,r}$ be a complete bipartite graph where the edges have i.i.d. weights and come from a distribution with mean $\mu$ that is supported on $[0,b]$; Assume that $ck\mu \geq 1+\epsilon$ for some $\epsilon > 0$. If the algorithm from Section \ref{fixed-degree} is used to sample a subgraph $H$ from $G$, then
\[ \E[S] = \sum_{v\in R} \E[X_v] = r\left(1-\exp\left(-\frac{2l\epsilon^2}{b^2}\right)\right) \]
\end{thm}

There are two things to note about this variant. The first is that
since the variables $X_v$ are negatively correlated, our results in
Subsection \ref{fixed-degree} can be readily extended to the results of this
section. The second is that the condition that $W_{uv}$ are i.i.d
is not necessary to obtain the full effect of the analysis. Indeed,
the only place in the proof where the fact that $W_{uv}$ are i.i.d
is when we argued that $X_{uv}$ is large with high probability by a
Hoeffding bound. For the bound to apply, it is sufficient to assume
that $W_{uv}$ for all $v$ are independent. In particular, it is
possible that $W_{uv}$ for all $u$ are inter-dependent. This allows
us to assume a weight distribution that depends on the strength of
the recommender and the relevance of the recommendation separately.
