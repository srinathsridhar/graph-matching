\subsection{The Greedy Algorithm}
\label{greedy}
The results in the previous section concentrated on producing nearly
optimal solutions in expectation. In this section, we will show that
it is possible to obtain good solutions regardless of the model that
generated the recommendation subgraph. \vs

We analyze the following natural greedy algorithm for constructing a $(c,a)$-recommendation
subgraph $H$ iteratively. \vspace{0.05in}

\begin{algorithm}[H]
  \SetAlgoLined
  \KwData{A bipartite graph $G=(L,R,E)$}
  \KwResult{A (c,a)-recommendation subgraph $H$}
  \For{$v$ in $R$}{
    freelinks \leftarrow $\{u \in N(v) | useddegree[u] < c\}$\;
    \If{length(freelinks) $\geq a$}{
      restrict freelinks to $a$ elements\;
      \For{$u$ in freelinks} {
        $H \leftarrow H \cup \{(u,v)\}$\;
        useddegree[$u$] $\leftarrow$ useddegree[$u$]+1\;
      }
   	}
  }
  \Return $H$\;
  \caption{The greedy Algorithm}
\end{algorithm}\vs

The algorithm loops through each vertex in $R$, and does a constant amount
of work in each iteration. Therefore, the runtime is linear. Furthermore,
the only data structure we use is an array which keeps track of $\deg_H(u)$
for each $u\in R$, so we only use linear memory as well. Finally, we prove
prove the following tight approximation property of this algorithm.

\begin{thm}
The greedy algorithm gives at last a $1/(a+1)$-approximation to the $(c,a)$-graph
recommendation problem.
\end{thm}
\begin{proof}
Let $R_{GREEDY}, R_{OPT}\subseteq R$ be the set of vertices that have
degree $\geq a$ in the greedy and optimal solutions respectively. Note
that any $v \in R_{OPT}$ along with neighbors $\{u_1,\ldots u_a\}$
forms a set of candidate edges that can be used by the greedy
algorithm.
%So we can consider $R_{OPT}$ as a candidate pool for $R_{GREEDY}$.
Each selection of the greedy algorithm might result in
some candidates becoming infeasible, but it can continue as long as the candidate pool is not depleted.
Each time the greedy algorithm selects some vertex $v\in
R$ with edges to $\{u_1,\ldots, u_a\}$, we remove $v$ from the candidate pool.
If any $u_i$ had degree $c$ in the optimal solution, we would also need to
remove an arbitrary vertex $v_i\in R$ adjacent to $u_i$ from the optimal
solution.
%In other words, by using an edge of $u_i$, we force it to
%not use an edge it used to some other $v_i$, which might cause the
%degree of $v_i$ to go below $a$.
Therefore, at each step of
the greedy algorithm, we may remove at most $a+1$ vertices from
the candidate pool as illustrated in Figure~\ref{fig:greedy}. Since our candidate pool has size $OPT$, the
greedy algorithm can not stop before it has added $OPT/(a+1)$
vertices to the solution.
\end{proof}

\begin{figure}[H]
\centering
\includegraphics[width=.39\textwidth]{images/greedy.png}
\caption{This diagram shows one step of the greedy algorithm. When $v$ selects edges to $u_1,\ldots, u_a$, it potentially removes $v_1,\ldots, v_a$ from the pool of candidates that are avaiable. The potentially invalidated edges are shown in red.}
\end{figure} 

This approximation guarantee is as good as we can expect, since for $a=1$ we recover the familiar
$1/2$-approximation of the greedy algorithm for matchings.
As in matchings, randomizing the order in
which the vertices are processed still leaves a constant factor gap
in the quality of the solution~\cite{KarpVaziraniVazirani1990}.
Despite this result, the greedy algorithm fares much better when we
give it the same expectation treatment we have given the sampling
algorithm. Switching to the Erd\"{o}s-Renyi model instead of the
fixed degree model used in the previous section, we now prove the
near optimality of the greedy algorithm for the $(c, a)$-recommendation
subgraph problem.

\begin{thm}
Let $G=(L,R,E)$ be a graph drawn from the $G_{l,r,p}$. If $S$ is the size of the $(c,a)$-recommendation subgraph produced by the greedy algorithm, then:
\[ \emph{\E}[S] \geq r - \frac{a(lp)^{a-1}}{(1-p)^a} \sum_{i=0}^{r-1}(1-p)^{l-\frac{ia}{c}}\]
\end{thm}
\begin{proof}
Note that if edges are generated uniformly, we can consider the
graph as being revealed to us one vertex at a time as the greedy
algorithm runs. In particular, consider the event $X_i$ that the
greedy algorithm matches the $(i+1)^{th}$ vertex it inspects. While,
$X_{i+1}$ is dependent on $X_1,\ldots, X_i$, the worst condition for
$X_{i+1}$ is when all the previous $i$ vertices were from the same
vertices in $L$, which are now not available for matching the
$(i+1)^{th}$ vertex. The maximum number of such invalidated vertices
is at most $\lceil i/c \rceil$. Therefore, the probability that fewer
than $a$ of the at least $l-\lceil i/c \rceil $ available
vertices have an edge to this vertex is at most $\Pr[Y\sim Bin(l-\frac{i}{c},p): Y < a]$.
We can bound this probability by bounding each term in its binomial
expansion by $(1-p)^{l-\frac{ia}{c}-a+1}(lp)^{a-1}$ to obtain the following.

\[ \Pr[Y\sim Bin(l-\frac{ia}{c},p): Y < a] \leq a (1-p)^{l-\frac{ia}{c}-a+1}(lp)^{a-1}\]

Summing over all the $X_i$ using the linearity of expectation and this upper bound,
we obtain
\begin{align*}
      \E[S]
&\geq r - \sum_{i=0}^{r-1} \E[\lnot X_i] \\
&\geq r - \sum_{i=0}^{r-1} \Pr[Y \sim Bin(l-\frac{ia}{c},p): Y < a] \\
&\geq r - a(lp)^{a-1}\sum_{i=0}^{r-1}(1-p)^{l-\frac{ia}{c}-a+1}
\end{align*}
\end{proof}

Asymptotically, this result explains why the greedy
algorithm does much better in expectation than $1/(a+1)$ guarantee we
can prove in the worst case. In particular, suppose $a$ and $c$ are fixed and that
$l/r$ is taken to be a constant as both $l$ and $r$ tend to $\infty$. In the realm where sublinear error is possible (i.e. when $lc/a>r$) each term in the sum above becomes $\Theta(l^{-\epsilon})$ for some $\epsilon>0$ if we set $p=\Theta(\log(l)/l)$. Consequently, the error term reduces to $\Theta(l^{1-\epsilon}\log^a(l))$ which is sublinear on the number of vertices.
